\documentclass{jfp1}

\usepackage{color}

\title{The GHC Runtime System}
\author{fill in}

\newcommand{\Red}[1]{{\color{red} #1}}
\newcommand{\ToDo}[1]{{\color{blue} ToDo: #1}}
\newcommand{\SM}[1]{{\color{green} SM: #1}}
\newcommand{\slpj}[1]{{\color{blue} SLPJ: #1}}
\newcommand{\nb}[1]{$\lhd$ \Red{#1} $\rhd$}
\newcommand{\tocite}[0]{{\color{red} [cite]}\xspace}

\newcommand{\XXX}{\Red{XXX}}

\begin{document}

\maketitle

\begin{abstract}
(This paper describes the implementation of the GHC runtime system)
\end{abstract}

\tableofcontents

\section{Introduction}

The GHC runtime system is perhaps one of the most unusual language
runtimes in wide use today.  Much of its implementation was directly
motivated by the unusual (by popular programming language standards!)
features in Haskell which the runtime needs to support.  Some of these
features, such as lazy evaluation and ubiquitous concurrency, complicate
the design of the system and require us to do a lot more work to give
programmers the high-level behavior they desire.  Other features, such
as an emphasis on pure computation without mutation, simplify the design
components such as garbage collection and software transactional memory.

% ALTERNATE OPEN
%   runtimes in wide use today.  Yes, at its very core, it contains many of
%   the features that programmers have come to expect in the runtime systems
%   of high-level programs: an implementation of exceptions, a generational
%   garbage collector and support for multithreaded execution.  But a large
%   portion of the runtime is dedicated to supporting features of Haskell
%   which are rarely found in other programming languages: lazy evaluation,
%   lightweight preemptive concurrency and software transactional memory, to
%   name a few.

%   Much of its implementation was directly
%   motivated by the unusual (by popular programming language standards!)
%   features in Haskell which require runtime support.  Some of these
%   features, e.g. lazy evaluation and ubiquitous concurrency, complicate
%   the design of the system and require us to do a lot more work to give
%   programmers the high-level behavior they desire.  Other features, such
%   as an emphasis on pure computation without mutation, simplify the design
%   components such as garbage collection and software transactional memory.

\Red{Complete, flesh out, rewrite}

\SM{This is a good start.  I think it will help to tell the reader up
  front that we intend to focus on the main novel aspects of the
  runtime how concurrency and parallelism, the garbage collector, and
  lazy evaluation work.  Everything else we say will be stuff that you
  need to know in order to understand these three main topics, and the
  order of the paper will reflect this - talking about the essential
  aspects of the evaluation model and data representations first,
  because you need that to understand the rest.}

\SM{Please feel free to label sections that you would like me to
  write.}

%\section{Execution model}

We begin our exploration of GHC's runtime system with a description
of the execution model of compiled Haskell code.

While the execution model of compiled code isn't in the domain of the
runtime system per se (it is the compiler's responsibility to generate
code that abides by the various conventions), various parts of the
execution model (e.g. heap representation, lazy evaluation) have an
important role to play in the design of components of the runtime system
(e.g. the garbage collector, concurrency).  In this section, we offer a
brief explanation of some of the most important aspects of the
\emph{Spineless Tagless G-machine} (STG), GHC's core execution model.

\subsection{STG}

How does a Haskell program run?  This is not a straightforward question
to answer, because a compiled Haskell program is not run directly: it is
desugared into an intermediate representation called Core, optimized,
and transformed into another intermediate representation (STG) before
any code is generated.  Describing how Haskell is desugared and
optimized is out of scope for this paper.  Thus, we cannot directly talk
about how Haskell runs, rather, we can only talk about a abstract
machine language which resembles Haskell, but is designed to be easy to
generate code for.

This abstract machine is the \emph{Spineless Tagless G-machine}, and it
is a functional language.  Like Haskell, it has function application,
literals, constructors, primitive operations, let-bindings amd
case-expressions.  However, it imposes a few simplifying restrictions
(for example, there are no lambda literals: they must be top-level or
let-bound before use).  These restrictions allow us to make four
important equivalences between language and operation:

\begin{itemize}
    \item \emph{Let expressions} are \emph{heap allocation},
    \item \emph{Function application} is a \emph{tail call},
    \item \emph{Case expressions} are \emph{evaluation}, and
    \item \emph{Constructor application} is a \emph{return to continuation}.
\end{itemize}

We now consider each of these constructs in turn.

\subsection{Let-binding and heap allocation}

Heap allocation is of central importance to Haskell programs: most data
values, function values, and thunks are all allocated on the heap.  These objects are
all referred to collectively as \emph{closures}.  \Red{Figure}

An important optimization is the \emph{let-no-escape} binding.  True
to its name, let-no-escape indicates that the contents of the variable
do not escape from the scope it was allocated in.  Because 

\subsection{Function application and tail calls}

%   \begin{itemize}
%       \item There are no literal lambdas; instead, a lambda must either be lifted to the top-level of the program or be allocated via a let-binding.  Requiring this makes \emph{heap allocation} explicit, because a heap allocation can only occur when a let-binding occurs.
%       \item Constructor application and primitive operations must be \emph{saturated}, that is, they must be provided with all of the arguments they expect.  Ordinarily, these operations are curried, so partially applied forms are OK; in STG, they are wrapped in lambdas.
%       \item Evaluation of expressions and case-split of the result,
%       \item Heap allocation (let) of closures (thunks or functions) and constructors,
%       \item Let-no-escape,
%   \end{itemize}


\subsection{(Untitled)}

The fact that all function applications are tail calls is perhaps what
causes the largest divergence between the runtime execution of Haskell
and other block-structured languages.  In C, when a function is entered
a stack frame (e.g. continuation) is always pushed onto the stack; in
Haskell, we may push zero, one, or more frames on the stack before
jumping to the function destination.

\subsection{Heap representation}

\subsection{Spineless: Stack}

\Red{Maybe defer this later?}

\Red{Haskell code executes utilizing a \emph{eval/apply} model}



\subsection{Tagless: Lazy evaluation}

Unusually, the info table also contains \emph{code} for the object: any
object can be ``entered'' (by jumping to the code in the infotable) in
order to evaluate it.  This is perhaps the most important design
decision in STG: it prescribes a uniform data representation for both
fully evaluated data values (constructors and functions) and unevaluated
thunks---referred to collectively as \emph{closures}.  In the original
design of the STG, all code wanting to access a field in a data
structure first entered the closure and upon return would receive the
values of the data structure (a \emph{vectored return}).  The lack of
any check whether or not a thunk was evaluated or not (an obvious
alternative implementation strategy) is behind the ``tagless`` in STG's
name.

\SM{We don't do vectored returns, since GHC 4.0. Also
  return-in-registers, which you might also find menioned in the
  original STG paper, was removed in favour of unboxed tuples.}

\SM{Tagless is a bit debatable these days.  We have tags in pointers
  (pointer tagging).  Also the code pointer for a function closure is
  the entry code for the function (not the ``eval'' code).}

The fact that objects are always entered is an extremely helpful layer
of indirection which is used for a variety of purposes by the runtime.
For example, when a thunk finishes evaluation, we need to write back the
true value to the old memory location so the computation is not
repeated.  If the new value is larger than the thunk, it must be placed
in new heap memory and the old thunk replaced with an \emph{indirection}
pointer pointing to the new value.  An indirection object, then, simply
jumps to the code of the indirectee!  This flexibility is useful for a
number of other exceptional conditions, especially in the case of
concurrency.

Unfortunately, code compiled this way performs a lot of indirect jumps,
and modern branch prediction units on processors typically perform very
poorly under such control flow.  This might suggest that this data
representation is quite expensive and of little interest to implementors
of non-lazy languages.  Fortunately, modern GHC implements a
\emph{dynamic pointer tagging} scheme~\XXX{} which, in many cases,
eliminates the need to perform an indirect jump.  This scheme works by
using the lower order bits (two bits in a 32-bit machine, and three bits
in a 64-bit machine) in order to encode whether or not pointer is
already evaluated, and if it is, what the tag of the constructor is.
When the tag bit is absent, user code will enter the closure, as before.

We think that this representation is well worth considering even for
non-lazy-by-default languages.  In the case of data types with multiple
constructors, pointer tagging enables us to support efficient case
analysis of user-defined types \emph{without a memory
dereference}---only the tag bit must be consulted.  And, to reiterate,
the added flexibility the indirect jumps will greatly aid us later in
other components of the runtime system.

\Red{Maybe explain in more detail}

\subsection{Black holes} \label{sec:blackhole}

One particular type of closure worth some attention is the \emph{black
hole}.  The semantics of a black hole are relatively simple: a black
hole represents a thunk that is currently being evaluated.  A thunk
can be \emph{claimed} by overwriting it with a black hole. The entry
code for a black hole should arrange for the thread to receive the value
of the thunk after it is evaluated, some way or another.

Black holes were originally proposed as a solution for a space leak that
occurs when tail calls are being made in the STG.~\cite{Jones2008} The
problem is relatively simple: suppose that you are evaluating the thunk \verb|last [1..10000]|,
where \verb|last| is a tail recursive function:

\begin{verbatim}
last []     = error "empty list"
last (x:[]) = x
last (x:xs) = last xs
\end{verbatim}

Because \verb|[1..10000]| is constructed lazily, you might expect
evaluating this thunk to take constant space, since the front of the
list is never retained.  However, there is a problem: the front of the
list is retained by the thunk itself, \verb|last [1..10000]|.  The thunk
will eventually get overwritten by the result of the computation, but
only at the \emph{end} of the computation, by which point the entirety
of \verb|[1..10000]| is resident in memory.  To solve this problem, when
\verb|last [1..10000]| is evaluated, we can \emph{eagerly} overwrite it
with a \verb|BLACKHOLE|; now the thunk no longer retains the list.  As
an optimization, we can \emph{lazily} blackhole by waiting until the
thread becomes descheduled, e.g. for a garbage collection.

The simplicity of black holes belies their utility in a variety of contexts.
In particular:

\begin{itemize}
\item Black holes allow us to detect some infinite loops: if a thread attempts
    to evaluate a black hole which it previously claimed, that is an infinite loop!
\item In a multiprocessor setting, black holes allow us to avoid duplicating work when multiple
    threads attempt to evaluate the same object.\footnote{It is worth noting that for this use-case, we have to \emph{eagerly} blackhole thunks.}  Instead, a thread blocks on the owner
    of a black hole, waiting for it to finish processing.  This is described in more
    detail in Section~\ref{sec:sync}.
\end{itemize}

\subsection{Notes}

\Red{More stuff}

While the original paper about the Spineless Tagless
G-machine~\cite{PeytonJones1992} remains the best source for an
in-detail explanation about STG, many details have changed over the two
decades it has been last published.  If one had to sum up its modern
implementation in GHC, one might call it the ``Usually-Spineless
Mostly-Tagged G-machine.''  Briefly, the important changes are as
follows:

\begin{itemize}
    \item STG is no longer compiled into C, instead, it is
        compiled into a more low-level language C--~\cite{Jones1999}, allowing
        GHC to change details such as calling conventions and utilize
        more low level functionality (e.g. tail calls, explicit stack
        layout, computing targets of jumps and implementing exceptions).
    \item The stipulation that closures have a uniform representation
        has been relaxed.  The most important change is how functions
        are represented: while the paper originally proposed a push/enter
        model, GHC now uses an eval/apply model~\cite{Marlow2006}, in
        which the code pointer of function info tables points to the code
        for the function itself, and not code for reading arguments off of
        the stack.  This also means that the stack representation is
        different: there is now only a single stack for continuations and
        update frames (alongside the second C stack for register spills).
    \item Furthermore, pointers are dynamically tagged with information
        that may allow code to avoid jumping into a closure in order to
        ensure that it is evaluated---STG is not
        tagless!~\cite{Marlow2007}  The original dynamic tagging paper
        suggests that these tags could be erased for only a performance
        hit, but in fact, in some places, they are required.  This also
        means that vectored-returns have gone the way of the dinosaur.
    \item The spineless in STG's name refers to the fact that GHC stores
        the intermediate state of evaluating a thunk on the stack,
        rather than on the ``spine'' of the thunk itself.  Under some
        circumstances, such as in the case of an interrupt, it is
        profitable to write out this state back to the thunk, so that
        this work can be resumed later.~\cite{Reid1999} \Red{Maybe this is gratuitous}
\end{itemize}

\section{Storage}

An essential component of a runtime system for any high-level
programming language is the garbage collector, which is responsible for
identifying and reclaiming memory from objects which are no longer in
use by the program.  When it comes to a garbage collector,
\emph{efficiency} is the order of the day: the speed of the garbage
collector affects the performance of all programs running on the
runtime, and thus the GHC runtime devotes a substantial portion of its
complexity budget to a fast garbage collector.  What do we need for
a fast garbage collector?

%   you should block structure your heap
%   garbage collection in GHC is cheaper than you thinko
%   produce a lot more garbage than imperative languages
%   the larger percent of your values are garbage, the faster it works
%   immutable data never points to younger values
%   gc is hard, but gc for immutable languages is much easier

%   .NET advice:
%       Allocate all of the memory (or as much as possible) to be used with a given data structure at the same time.
%       Remove temporary allocations that can be avoided with little penalty in complexity.
%       Minimize the number of times object pointers get written, especially those writes made to older objects.
%       Reduce the density of pointers in your data structures.
%       Make limited use of finalizers, and then only on "leaf" objects, as much as possible. Break objects if necessary to help with this.

\subsection{Blocks}

The very first consideration is a low-level one: ``Where is the memory coming
from?''  The runtime can request memory from the operating system via
\verb|malloc|, but how much should it request, and how should it be
used?  A simple design is to request a large, contiguous block of memory
and use it as the heap, letting the garbage collector manage objects allocated
within it.  However, this scheme is fairly inflexible: when one of these heaps
runs out of memory, we need to double the size of the heap and copy all of the
old data into the new memory.  Picking the initial sizes of heaps can be an exercise
in fiddling with black magic tuning parameters.

A more flexible scheme is a \emph{block-structured heap}~\cite{maclisp,Dybvig94don'tstop,Marlow:2008:PGG:1375634.1375637}.
The basic idea is to divide the heap into fixed-size $B$-byte blocks,
where $B$ is a power of two: blocks are then linked together into chains in order to
provide memory for the heap.\footnote{GHC uses 4kb blocks, but this is an easily
adjustable constant.}  These blocks need not be contiguous: thus, if your
heap runs out of space, instead of having to double the size of your heap,
you can simply chain a few more blocks onto it.  There are some other benefits as well:

\begin{enumerate}
    \item Large objects (i.e., block sized or larger) do not have to be copied from one region to
        another; instead, the block they reside in can be relinked from
        one region to another.\footnote{Of course, this requires \emph{only}
        one object to live in a block, which can result in fragmentation.
        However, empirically this does not seem to have caused much of a problem.}
    \item Free memory can be recycled quickly, since a free block can be quickly
        reused somewhere else.
    \item Blocks make it easy to provide heap memory in contexts where it is
        not possible to perform garbage collection.  As an example, consider
        the GMP arbitrary-precision arithmetic library.  This C code requires
        the ability to allocate memory while performing internal computation.
        However, if the heap runs out of memory, what can you do?  If the heap
        were contiguous, you would now need to carry out a GC to free up some memory
        (or get a larger heap); but this would require us to halt the C code
        (arbitrarily deep in some internal computation) while simultaneously being
        able to identify all pointers to the heap that it may be holding.  Whereas
        in a block-structured heap, we can simply grab a new block and defer the GC
        until later.
\end{enumerate}

One reason why this scheme works so well is that most objects on the
heap are much smaller than the block size, so handling these cases is very
simple.  When an object is larger than a block size, it needs to be
placed into a \emph{block group} of contiguous blocks---which in turn
need to be handled with some care to avoid fragmentation.  The blocks
themselves are provided by the operating system in large units called
\emph{megablocks}.\footnote{1Mb in size, in the current implementation.}

Finally, each block is associated with a \emph{block descriptor}, which
contains information about the block such as what generation it belongs to, how full it is, what block
group it is part of, and so on.  Finding the block descriptor
associated with a given address in memory is a very common
operation---the garbage collector needs to do this for every object it
visits, for example.  Hence the placement of the block descriptor is
designed such that the block descriptor for a given address is a pure function
of the address, and can be computed in a few instructions.

An obvious place to put the block descriptor is at the beginning of a
block, but this runs into problems when the block is a member of a
block group (the memory must be contiguous!).  Thus, the descriptors
of blocks of a megablock are instead organized together at the
beginning of a megablock, and megablocks themselves are required to be
megablock-aligned in memory.

\subsection{Memory layout}

Before we can discuss the garbage collector proper, we have to describe
the layout of the data that is to be garbage collector.  GHC has a uniform
representation for objects on the heap with a \emph{header}, which indicates
what kind of object the data is, and a \emph{payload}, which contains
the actual data for an object (e.g. free variables for function values
and thunks, or fields for data values).  The header points to an
\emph{info table}, which provides more information about what kind of
object the closure is, what code is associated with the object and what
the layout of the payload is (e.g. what fields are pointers.)

The presence of info tables makes it easy for the garbage collector to
determine what other closures on the heap an object may reference, as it
says which fields in the payload are pointers.  Essentially everything
that the GC touches has an info table, including the stack frames (each
block of compiled code receives its own info table.)  In particular,
this makes calculating the \emph{roots} (base objects which are always
considered reachable) of the application much simpler: at the beginning
of any block of code, the info table and registers (which known and
saved by the code itself) constitute all of the pointers in use,
allowing us to accurately perform GC.

There are many possible methods by which objects on the heap can be
represented (for example, in place of info tables, pointer tagging can
be used to distinguish non-pointers from pointers).  However, in order
to support lazy evaluation, headers have
another important function, which is that they double as pointers to the
\emph{entry code} responsible for evaluating a thunk.  The ability to \emph{replace}
one header with another and have the behavior of a thunk change
correspondingly is tremendously useful.  The issues are discussed in Section~\ref{sec:lazy}.

\subsection{Generational garbage collection}

The next question you might ask is, ``What kind of garbage collector
should I use?''  By default, GHC uses a generational copying collector.

A \emph{generational collector} divides the heap into
\emph{generations}, where generations are numbered with the zero being
the youngest.  Objects are allocated into the youngest generation, which
has garbage collection performed on it whenever it runs out of memory.
Surviving objects are \emph{promoted} to the next generation, which is
collected less frequently, and so forth.  In a \emph{copying collector}, this
promotion occurs by simply copying the object into the \emph{to-space},
a process called \emph{evacuation}.  Evacuated objects are subsequently
\emph{scavenged} to find what objects they refer to, so they can be evacuated as well.

%   \SM{We probably shouldn't use the terms ``evacuate'' and
%     ``scavenge'', since I think the GHC GC uses them inconsistently with
%     the literature. (TODO: check).}

%   \Red{EZY: I think, probably, we should just use these terms, since the literature
%   doesn't really have a snappy name for "copy to to-space" and "scan for pointers",
%   and I use these terms a lot later on.}

The efficacy of generational collection hinges on the ``generational
hypothesis'', which states that data that has been recently allocated is
the most likely to die and become unreachable.  This tends to be
particularly true of functional programs, which encourage the use of short-lived
intermediate data structures to help structure computation.  In fact,
functional programs allocate so much memory that it makes sense not to
immediately promote data, since objects may not have had sufficient
chance to die by the time of the first GC.  Thus, GHC further implements
an \emph{aging} scheme, where reachable objects in generation 0 are not
immediately promoted to generation 1; instead, they are aged and
promoted the next GC cycle.\footnote{When objects are only aged once, an
equivalent way of stating this scheme is that generation 0 is split into
two generations, but we never garbage collect just the younger
generation, we always collect both on a minor collection.  This is in fact
how GHC implements aging.}

Use of a copying collector has other benefits for allocating heavy
workloads.  In particular, copying collection ensures that free memory
is contiguous, which allows for extremely efficient memory allocation
using a \emph{bump allocator}---so named because a heap allocation
simply involves bumping up the free space pointer. Additionally, while
copying collectors are often criticized for wasting half of their
allocated memory to maintain the two spaces for copying, a block
structured heap can immediately reuse blocks in the from-space as soon
as they are fully evacuated.

\subsubsection{Mutability in the GC}

The primary complication when implementing a generational garbage
collector is the treatment of mutable references.  When a heap is
immutable, pointers in the young generation can only ever point into
older generations; thus, to discover all reachable objects when
collecting an old generation, it suffices to simply collect all younger
generations when performing an (infrequent) collection of an older
generation.  However, if objects in the old generation are mutable, they
may point back into the young generation, in which case we need to know
that those objects are reachable even when the only references to them
are in the old generation (which we would like to avoid collecting).

The solution to this problem is to apply a \emph{GC write barrier}
(sometimes confusingly referred to as a \emph{write barrier}) to memory
writes, adding the mutated object to a \emph{remembered set} which is
also considered a root for garbage collection.  Now the GC cannot
accidentally conclude an object pointed to by a mutable reference in an
old generation is dead: it will discover its reachability through the
remembered set.  However, this scheme is costly in two ways: first, all
mutation must pay the overhead of adding the object to the remembered
set, and second, as the remembered set increases in size, the amount of
heap that must be traversed during a minor collection also increases.

In the first case, GHC keeps track of mutation per object, spending a
single memory write to add a mutated object to a mutable list.  This
design lies in a continuum of precision: one could increase the
precision of the remembered set by only adding mutable fields (rather
than objects), or one could decrease the precision by only tracking
\emph{cards} (i.e. portions of the heap) at a larger granularity.
Increased precision increases the overhead of the mutable list but
reduces the amount of extra work the GC needs to perform, while reduced
precision makes mutation more efficient but leads to slower minor
collections.  We think that mutation per object is a good balance: mutation
is not prevalant enough in functional code that coarse-grained card making
buys much, and most mutable objects in Haskell are quite small, with only
one or two fields.\footnote{However, this assumption has caused the GHC runtime
some grief, e.g. in the case of mutable arrays of pointers, which we used to
scan the entirety.  Today, we have a card-marking scheme to permit mutable
arrays to be efficiently GC'd.}

In the second case, GHC can take advantage of an interesting property of
lazy functional programs: thunks are only ever mutated once, in order to
update them with their fully evaluated values---they are immutable
afterwards.  Thus, we can immediately eliminate an updated thunk from
the mutable list by \emph{eagerly promoting} the data the updated thunk
points to into the same generation as the thunk itself.  Since the thunk
is immutable, this data is guaranteed not to be GC'd until the thunk
itself is GC'd as well.  This leads to an interesting constraint on how
garbage collection proceeds: we must collect older generations first, so
that objects we may want to promote have not been evacuated yet.
Because an already evacuated object may have forwarding pointers
pointing to it, it cannot be evacuated again within the same GC.

%   This procedure is divided into two steps: \emph{evacuation},
%   which performs the copy, replacing the old object with a forwarding
%   pointer to the new object, and \emph{scavenging}, which scans all
%   objects that are newly copied in to-space and evacuates their fields.

% \SM{You captured all the issues here well. Nice job!}

\subsection{Parallel garbage collection}
\label{sec:parallel-gc}

A generational garbage collector offers quite good performance, but
there is still the question, ``Is it fast enough?''  One avenue for
speeding up the garbage collector when multiple cores are available is
to perform \emph{parallel garbage collection}, having multiple threads
traverse the heap in parallel.  Note the distinction from
\emph{concurrent collection}, where the GC runs concurrently with user
code which is mutating values on the heap.  GHC implements parallel
collection~\cite{Marlow:2008:PGG:1375634.1375637} but not concurrent
collection: concurrent collection requires synchronization between the
GC and the mutator and consequently is more complex.  However, we have
experimented with a form of concurrent collection in which individual
cores have local heaps that can be collected independently of activity
on the other cores~\cite{local-heaps}.

There are two primary technical challenges that accompany building a
parallel garbage collector.  The first is how to divide the GC work
among the threads, the second is how to synchronize when two GC threads
attempt to evacuate the same object.

GHC overcomes the first challenge by utilizing the block structure of
the heap.  In particular, a block in the to-space of a garbage
collection constitutes a unit of work: a thread can either claim the
block to scavenge for itself, or the block can be transferred to another
thread to process.  Once blocks are chosen as the basic unit of work,
there are a variety of mechanisms by which work can be shared: blocks
can be statically assigned to GC threads with no runtime load balancing,
blocks can be taken from a global queue which provides blocks to all
threads, or a hybrid solution can have threads have local queues, but
permit other threads to steal work from other queues when they are idle
via a \emph{work-stealing queue
structure}.~\cite{Arora:1998:TSM:277651.277678}  GHC originally
implemented a single global queue, but we have since switched
work-stealing queues because they have much better data locality, as
processors prefer to take work from their local queues before stealing
work from others.  In fact, we don't want to do any load-balancing on
minor collections, because it ruins locality by shipping work off to
another core when the data is likely \emph{already} in the cache of the
original core.~\cite{Marlow2009}

The second challenge reflects the primary cost of parallel GC, which is
the extra synchronization overhead any parallel scheme will impose.  In
particular, we must prevent the GC from duplicating mutable objects when
multiple threads attempt to evacuate the same object by synchronizing
the object (by either locking it pessimistically or compare-and-swapping
optimistically).  This synchronization is expensive; fortunately, there
is a wonderful benefit for immutable objects: they require no
synchronization, because it is safe to have multiple copies of an
immutable data structure!  Eliminating the locks in these cases accounts
for a 20-30\% speedup, which is nothing to sneeze at.

One important parameter which must be set properly is the size of the
young generation, i.e. the nursery.  If the nursery is too small, then
we will need to perform minor garbage collections too frequently.  But
if it is too large, then cores will generate a lot of memory traffic
getting data that is not in their cache.  In general, memory bandwidth
is the bottleneck when multiple cores are allocating quickly; thus,
having the nursery be the size of the cache is generally the best setting.
\Red{But perhaps see the discussion at } \verb|http://donsbot.wordpress.com/2010/07/05/ghc-gc-tune-tuning-haskell-gc-settings-for-fun-and-profit/|

%   In general, writing a parallel garbage collector is tricky business:
%   it's easy to wipe out performance gains by accident.  Of particular delicacy
%   is the cache behavior

%   \SM{I think we don't need to talk about the two types of locking (I
%     don't think there's a deep reason to choose one over the other, just
%     minor engineering or performance issues).  What's more important to
%     mention here is that we can avoid locking immutable objects---a win
%     for immutability.}

%   \SM{I think we should talk about the whole idea of having small
%     nurseries that fit in the cache, since this is the only way to scale
%     GC when you have multiple cores allocating fast (they run into the
%     memory bandwidth otherwise).  And related to this is the idea that
%     we don't want to do any load-balancing when we do G0 collections,
%     because that ruins the cache.  Some measurements in the ICFP'09
%     paper.}

\subsection{Summary}

\SM{Not sure whether going into this much detail is helpful to the
  average reader.  I'm prepared to be persuaded otherwise.}

We conclude by sketching the overall operation of GHC's parallel,
generational, block-structured garbage collector, with all of the
features that we have described thus far.

The main garbage collection function
\verb|GarbageCollect| in \verb|rts/sm/GC.c| proceeds (after all user
execution is halted) as follows:

\begin{enumerate}
    \item Prepare all of the collected generations by moving their blocks
        into the from-space and throwing out their mutable lists (recall the remembered
        set is only necessary when the generation is not being collected.)
        The blocks themselves indicate what generation live objects in them should be promoted to.
    \item Wakeup the GC threads, initializing them with eager promotion enabled.
    \item \emph{Evacuate} the roots of application (including the mutable lists of all older generations), giving work to the main GC thread to do.
    \item In a loop, each thread:
        \begin{enumerate}
            \item Look for local blocks to \emph{scavenge} (e.g. if the thread recently evacuated some objects which it hasn't scavenged), starting with blocks from the oldest generation.
            \item Try to steal blocks from another thread (at the very beginning of a GC, idle GC threads are likely to steal work from the main thread, if they didn't have any work to begin with).
            \item Halt execution, but while there are still GC threads running, poll to see whether or not there is any work to do.
        \end{enumerate}
    \item Cleanup after the GC, which includes running finalizers, returning memory to the operating system, resurrecting threads~\XXX, etc.
\end{enumerate}

The \emph{evacuate} function \verb|evacuate| in \verb|rts/sm/Evac.c|
takes a pointer to an object and attempts to copy it into a destination
generation to-space, as specified by the block it resides in (or the generation that
it needs to be promoted to, if eager promotion is enabled).  Before doing so,
it performs the following checks:

\begin{enumerate}
    \item Is the object heap allocated? (If not, it is handled specially.)
    \item Was the object already evacuated (e.g. the pointer already points
        to a to-space, or the object is a forwarding pointer)?  If it
        was, and to a generation which is younger than the intended
        target, then it reports the evacuation as failed (so scavenge
        can add a mutable reference pointing to the object to a mutable
        list, etc.)
\end{enumerate}

After the copy, the original is overwritten with a forwarding pointer.
If the object in question is mutable, this is done atomically with a
compare-and-swap to avoid races between two threads evacuating the same
object.

The \emph{scavenge} function \verb|scavenge_block| in
\verb|rts/sm/Scav.c| walks a pointer down the provided block (filled in
previously by evacuate), reading the info table in order to determine
what kind of object it is.  It evacuates the fields of the object,
temporarily turning off eager promotion if the object is mutable.  If
evacuation is unsuccessful for the field of a mutable object, it must be
added back to the mutable list.  When the block is finished being
scavenged, it gets pushed to the list of completed blocks.  The block
that is scavenged can be thought of as the ``pending work queue''; this
optimization was first suggested as part of Cheney's algorithm and
avoids the need for an explicit queue of pending objects to scavenge.

\subsection{Further reading}

While we have discussed many of the most important features of GHC's
garbage collector, there remain many other features we have not
discussed here.  These include:

\begin{itemize}
    \item an implementation of a compacting collector, \Red{no docs about this!}
    \item support for weak pointers and finalizers,~\cite{PeytonJones:1999:SSM:647978.743369} \Red{We might actually want to talk about this, it is probably one of the more voodoo-y parts of the system} and
    \item garbage collection of static objects.
\end{itemize}

We have also omitted many details about the features we have discussed.
For a good account of the block-structured parallel garbage collector,
please see~\cite{Marlow:2008:PGG:1375634.1375637}; however, since the
paper was published the default locking and load balancing schemes for
the parallel GC have changed, and we have implemented the improvement
described in Section 7.1.  Additionally, the GHC Commentary~\cite{ghc-gc-commentary} has good
articles for technically inclined GHC hackers on a variety of issues we
have discussed here, including eager promotion, remembered
sets \Red{ete etc}

\section{Concurrency and parallelism}

We now turn our attention to the implementation of
concurrency~\cite{PeytonJones:1996:CH:237721.237794} and
parallelism~\cite{Harris:2005:HSM:1088348.1088354} in the GHC runtime.
It is well worth noting the difference between concurrency and
parallelism: a parallel program uses multiple processing cores in order
to speed up computation, while a concurrent program simply involves
multiple threads of control, which notionally execute ``at the same
time'' but may be implemented merely by interleaving their execution.

GHC is both concurrent and parallel, but many of the features we
describe are applicable in non-parallel but concurrent systems (e.g.
systems which employ cooperative threading on one core): indeed, some
were developed before GHC had a shared memory multi-processor
implementation.  Thus, in the first section, we consider how to
implement \emph{threads} without relying on hardware support.  We then
describe a number of inter-thread communication mechanisms which are
useful in concurrent programs (and say how to synchronize them). Finally
describe what needs to be done to make \emph{compiled} code thread-safe.

\subsection{Threads}

Concurrent Haskell~\cite{PeytonJones:1996:CH:237721.237794} exposes the
abstraction of a \emph{Haskell thread}, which is a concurrent thread of
execution.  Given that operating systems also native threads, the
essential question for an implementation of Haskell threads is this:
what is the correspondence between Haskell threads and OS threads?

A simple scheme is to require a \emph{one-to-one} mapping, an approach
taken by many other languages with multithreading support.
Unfortunately, this approach is expensive: the GHC runtime would like to
support thousands of threads, which is supported poorly by most
underlying operating systems.  Furthermore, in the case that a program
is run with only one operating system thread (e.g. the code in question
is not thread safe), only one thread of execution is supported: all
concurrency is lost and must be implemented in userspace, as is the
case with many asynchronous IO libraries.

Instead, we must \emph{multiplex} multiple Haskell threads onto a single
OS thread.  This requires two adjustments: first, we must be able to
induce compiled Haskell code to yield to a scheduler, so that another
Haskell thread can be run.  Preemptive scheduling can sometimes be
difficult to implement, but for GC'd languages, there is an obvious
point where all threads get preempted: when they require a garbage
collection. Thus, we can preempt a thread by setting its heap limit to
zero, triggering a faux ``heap overflow'' that transfers control to the
scheduler.\footnote{This can have problems when a thread is in a tight,
    non-allocating loop; however, better concurrency in such cases can
be achieved by forcing the compiler to emit heap checks at all function
prologues, even when no allocation is necessary.}
Second, we must save the thread's context, e.g. the stack pointer and
the errno flag, so that it can be restored when the thread is
rescheduled.  Compiled code can be very efficient about how much other processor
state must be saved, because it knows what its state is.
While general purpose C-based coroutine libraries
must save all registers, a Haskell thread will usually only need
one or two registers to be saved.\footnote{This is what the fast path \texttt{stg\_gc\_*} functions are for.}

Once we have a way of suspending and resuming threads, the scheduler
loop is quite simple: maintain a run queue of threads, and repeatedly:

\begin{enumerate}
    \item Pop the next thread off of the run queue,
    \item Run the thread until it yields or is preempted,
    \item Check why the thread exited:
        \begin{itemize}
            \item If it ran out of heap, call the GC and then run the thread again;
            \item If it ran out of stack, enlarge the stack and then run the thread again;
            \item If it was preempted, the thread is pushed to the end of the queue;
            \item If the thread exited, continue.
        \end{itemize}
\end{enumerate}

Once we have a scheduler loop, we have a simple way of running multiple
Haskell threads on a \emph{single} OS thread, even when the Haskell
threads are not thread-safe.

\subsection{Foreign function interface}

While multiplexing is a very attractive solution for supporting
light-weight concurrency, there are some places when the illusion of ``a
Haskell thread is simply an operating system thread, but cheaper''
breaks down.  These cases are especially prevalent when considered with
the \emph{foreign function interface} (FFI)~\cite{Marlow04extendingthe},
which permits Haskell code to call out to foreign code not compiled by
GHC, and vice versa.  One particular problem is as follows: what if an
FFI call blocks?  As our concurrency is purely cooperative, if the FFI
call refuses to return to the scheduler, the execution of all other
threads will grind to a halt.

This problem requires us to decouple OS threads into two parts: the OS
thread itself (called a \emph{Task} in GHC terminology), and the Haskell
execution context (called a \emph{Capability}).  The Haskell execution
context can be thought of as the scheduler loop and is responsible for
the contents of the run queue: when executing, it is owned by the
particular OS thread which is running it.  A single-threaded Haskell
program would have one capability---in this case, the
capability can be thought as a global lock on the Haskell runtime.

Decoupling OS threads in this way allows a capability to be run on
different operating system threads: a blocking FFI call is handled by
releasing the capability before making the FFI call: if there is another
idle \emph{worker thread}, it can acquire the now free capability and
continue running Haskell code.  Why not move the FFI call rather than
the Haskell execution context?  The key is that we would like to avoid
an OS level context switch: from the perspective of the OS thread, it
executes an FFI call by releasing a lock (the capability), running the C
code and then re-acquiring the lock. \Red{Make sure this is right.}

Thus, we have to modify the scheduler loop as follows:

\begin{enumerate}
    \item Check if we need to yield the capability to some other OS thread, e.g. if an FFI call has just finished,
    \item \emph{Run as before.}
\end{enumerate}

Another problem occurs when the foreign code requires thread local
state.  Now that capabilities are passed around OS threads, we make no
guarantee that any given FFI call will be performed on the same OS
thread.  To accomodate this, Haskell introduces the concept of a
\emph{bound thread}, which is a thread which always runs on the same
operating system thread.  GHC also calls these \emph{in-calls}, due to
the fact that external code which calls into Haskell must be bound: if
it makes the Haskell code calls out via the FFI again, the inner and
outer C code may rely on the same thread local state.

How can we support bound threads?  A simple scheme is to give a bound thread
an OS thread and forbid any other thread from using it. However, we can
do better, at the cost of a little more complexity:

\begin{enumerate}
    \item \emph{Check if we need to yield the capability to some other OS thread, e.g. if an FFI call has just finished,}
    \item \emph{Pop the next thread off of the run queue,}
    \item Check if the thread is bound:
        \begin{itemize}
            \item If the thread is bound but is already scheduled on the OS thread, proceed.
            \item If the thread is bound but on the wrong OS thread, give the capability to the correct task.
            \item If the thread is not bound but this OS thread is bound, give up the capability, so that any capability that truly needs this OS thread will be able to get it.
        \end{itemize}
    \item \emph{Run as before.}
\end{enumerate}

While the dance of capabilities from task to task is somewhat intricate,
it imposes no overhead when bound threads are not used.

\subsection{Load balancing}

Assuming that the compiled Haskell code is thread safe~(see Section~\ref{sec:sync}), it is now
very simple to have the scheduler loop run on multiple OS threads:
allocate multiple capabilities!  Each OS thread in possession of a capability
runs the scheduler loop, and everything works the way you'd expect.

There is one primary design choice: should each capability have its own
run queue, or should there be a single global run queue?  A global run
queue avoids the need for any sort of load balancing, but requires
synchronization and makes it difficult to keep Haskell threads running on
the same core, destroying data locality.  With separate run queues, however,
threads must be load balanced: one capability could accumulate too many
threads while the other capabilities idle.

The very simple load balancing scheme GHC uses is as follows: when a
capability runs out of threads to run, it suspends itself (releasing its
lock) and waits on a condition variable.  When a capability has too many
threads to run (it checks each iteration of its schedule loop), it takes
out locks on as many idle capabilities as it can and pushes its excess
threads onto their run queues.  Finally, it releases the locks and
signals on each idle capabilities that they can start running.  The
benefit of this scheme is that the run queues are kept completely
unsynchronized, but a plausible alternative is to use work-stealing
queues.

\subsection{Sparks}

Up until now, the threads we have discussed were explicitly asked for
by the user.  In some cases, the user will only have one thread (because
their application does not need to be concurrent), in which case extra
cores will be unused.  Is there any way we can take advantage of extra
capacity?

Haskell takes advantage of pure lazy evaluation to offer \emph{sparks},
which are extremely light-weight threads with the express purpose of
evaluating a thunk to head normal form.  Because thunks do not have
side effects, it is safe to evaluate them speculatively: a spark may
or may not be evaluated, depending on whether or not there are idle capabilities.

Since sparks are even lighter-weight than threads, they are represented
separately and stored in \emph{spark pools}.  When a capability decides
it has no work to do, it creates a \emph{spark thread}, which repeatedly
attempts to retrieve a spark from the capability's spark pool and
evaluate it.  If the capability receives any real work (e.g. a thread
unblocks), it immediately declares that there are no more sparks to
evaluate so that the spark thread exits.

Whereas threads rarely need to be load balanced, sparks frequently need
to be migrated, as the capability that is generating
sparks is the one that also is doing real work.  Sparks are balanced using
bounded work-stealing queues~\XXX, where oldest sparks are stolen first.
Additionally, many sparks will end up never being evaluated: these sparks
are \emph{fizzled} and should be garbage collected. \Red{Indeed, the garbage
collector is responsible for taking fizzled sparks and removing them from the
spark pool... how does this work}

\subsection{MVars}

Haskell offers a variety of ways for Haskell threads to interact with
each other (MVars, asynchronous exceptions, STM, even lazy evaluation).
We now describe how to implement MVars, the simplest form of
synchronized communication available to Haskell threads.  An MVar is a mutable
location that may be empty.  There are two operations which operate on
an MVar: \verb|takeMVar|, which blocks until the location is non-empty,
then reads and returns the value, leaving the location empty, and
\verb|putMVar|, which dually blocks until the location is empty, then
writes its value into location.  An MVar can be thought of as a lock when
its contents are ignored.

The blocking behavior is the most interesting aspect of MVars:
ordinarily, one would have to implement this functionality using a
condition variable.  However, because our Haskell threads are not
operating system threads, we can do something much more light-weight:
when a thread realizes it needs to block, it simply adds itself to a
\emph{blocked threads queue} corresponding to the MVar.  When another
thread fills in the MVar, it can check if there is a thread on the
blocked list and wake it up immediately.

This scheme has a number of good properties.  First, it allows us
to implement efficient \emph{single wake-up} on MVars, where only one of
the blocking threads is permitted to proceed. Second, using a FIFO
queue, we can offer a fairness guarantee, which is that no thread
remains blocked on an MVar indefinitely unless another thread holds the
MVar indefinitely.  Finally, because threads are garbage collected
objects, if the MVar a thread is blocking on becomes unreachable,
\emph{so does the thread.}  Thus, in some cases, we can tell when
a blocked thread is deadlocked and terminate it.

\subsection{Asynchronous exceptions}

MVars are a cooperative form of communication, where a thread must
explicitly opt-in to receive messages.  Asynchronous
exceptions~\cite{Marlow:2001:AEH:378795.378858}, on the other hand,
permit threads to induce an exception in another thread
without its cooperation.  Asynchronous exceptions are much more
difficult to program with than their synchronous brethren: as a signal can
occur at any point in a program's execution, the program must be careful
to register handlers which ensure that any resources are released and
invariants are preserved.  In \emph{pure} functional programs, this requirement
is easier to fulfill, as pure code can always be safely aborted.  When
asynchronous exceptions are available, however, they are useful
in a variety of cases, including timing out long running computation,
aborting speculative computation and handling user interrupts.

Triggering an asynchronous exception is relatively simple with
preemptive scheduling: preempt the target thread back to the scheduler,
at which point the scheduler can introduce the exception and walk up the
stack, much in the same way a normal exception would be handled.  In
case a thread is operating in a sensitive region, an exception masking
flag can be set, which defers the delivery of the exception (it is saved
to a list of waiting exceptions on the thread itself).

There are two primary differences between how asynchronous exceptions
and normal exceptions are handled.  The first is that a thread which is
messaged may have been blocking on some other thread (i.e. on an MVar);
thus, when an asynchronous exception is received, the thread must remove
itself from the blocked list of threads.\footnote{If your queues are
    singly linked, you will need some cleverness to entries. GHC does
    this by stubbing out an entry with an indirection, the very same
that is used when a thunk is replaced with its true value, and modifying
queue handling code to skip over indirections; because blocking queues
live on the heap, the garbage collector will clean it up for us in the end.}

The second difference is how lazy evaluation is handled. When pure code
raises an exception, referential transparency demands that any other
execution of that code will result in the same exception.  Thus, while
we are walking up the stack, when we see update frames \Red{link back},
we replace the thunk it would have updated with a new thunk that always
the exception.  However, in the case of an asynchronous exception, the
code could have simply been unlucky: when someone else asks for the same
computation, we should simply resume where we left off.  Thus, we
instead \emph{freeze} the state of evaluation by saving the current
stack into the thunk.~\cite{Reid1999}  This involves walking up the stack
and perform the following operations when we encounter an update frame:

\begin{enumerate}
    \item Allocating a new \verb|AP_STACK| closure which contains the
        contents of the stack above the frame, and overwriting the
        old thunk\footnote{Possibly a black hole.} with a pointer to this closure,
    \item Truncate the stack up to and including the update frame, and
    \item Pushing a pointer to the new \verb|AP_STACK| closure to the stack.
\end{enumerate}

When another thread evaluates one of these suspended thunks,
\verb|AP_STACK| pushes the frozen stack onto the current stack, thus
resuming the computation.  The top of the new stack may point to
yet another suspended thunk, thus repeating the process until we
have reached the point of the original execution. \Red{This is probably not understandable}

\subsection{STM}

Software Transactional Memory, or STM, is an abstraction for concurrent
communication which emphasizes transactions as the basic unit of
communication.  The big benefit of STM over MVars is that they are
composable: while programs involving multiple MVars must be very careful
to avoid deadlock, programs using STM can be composed together
effortlessly.

Before discussing what is needed to support STM in the runtime system,
it is worth mentioning what we do not have to do.  In many languages,
an STM implementation must also manage all side-effects that any code
in a transaction may perform.  In an impure language, there may be many
of these side-effects (even if they are local), and the runtime must
make them transactional at great cost.  In Haskell, however, the type system
enforces that code running in an STM transaction will only ever perform
pure computation or explicit manipulation of shared state.  This eliminates
a large inefficiency that plagues many other STM implementation.

\Red{How is it implemented}

\Red{Maybe move this below messages and white holes}

\subsection{Messages and white holes}

In the descriptions above, we said very little about the synchronization
that is necessary to implement them in a multiprocessor environment.
Under the hood, the GHC runtime has two primary methods of synchronization:
\emph{messages} and \emph{white holes} (effectively a spinlock).  The
runtime makes very sparing use of OS level condition variables and
mutexes, since they tend to be expensive.

GHC uses a very simple message passing architecture to pass messages
between capabilities.  A capability sends a message by:

\begin{enumerate}
    \item Allocating a message object on the heap;
    \item Taking out a lock on the message inbox of the destination capability;
    \item Appending the message onto the inbox;
    \item Interrupting the capability, using the same mechanism as the context switch timer (setting the heap limit to zero); and
    \item Releasing the lock.
\end{enumerate}

This allows the message to be handled by the destination capability at
its convenience, i.e. after the running Haskell code yields and we
return to the scheduling loop.  In general, the benefit of message
passing systems is that they remove the need for synchronizing any of
the non-local state that another capability \emph{might} want to modify: instead,
the capability just sends a message asking for the state to be modified.

When sending a message is not appropriate, e.g. in the case of
synchronized access to closures which are not owned by any capability in
particular, GHC instead uses a very simple spinlock on the closure
\emph{header}, replacing the header with a \emph{white hole} header that
indicates the object is locked.  If another thread enters the closure,
they will spinlock until the original header is restored.  A spinlock is
used as the critical regions they protect tend to be very short, and it would
be expensive to allocate a mutex for every closure that needed one.

We can now describe how MVars and asynchronous exceptions are
synchronized.  An MVar uses a white hole on the MVar itself to protect
manipulations of the blocked thread queue; additionally, when it needs
to wakeup a thread, it may need to send a message to the capability
which owns the unblocked thread.  An asynchronous exception is even
simpler: it is simply a message to the capability which owns the thread.

\subsection{Synchronization and black holes} \label{sec:sync}

In previous sections, we asserted that execution of Haskell threads
could be made parallel, assuming that the compiled Haskell code was
thread-safe.  As any developer who has needed to write thread-safe code
can attest, this is a tall order!

Fortunately, much of this work is already done.  We have already seen various mechanisms
for explicit interthread communication, which are all designed with
concurrent execution in mind:  synchronizing them is a relatively simple
task.  Furthermore, the vast majority of Haskell code is pure, and needs no
synchronization.

However, there is one major feature we have to worry about: lazy
evaluation.  Recall that after a thunk has been evaluated, its value on
the heap must be replaced with the fully evaluated value. Multiple
threads could evaluate a thunk in parallel, so na\"ively, these writes
must be synchronized.  This is extremely costly: Haskell programs do a
lot of thunk updates!

Once again, our saving grace is purity: as thunks represent pure
computation, evaluating a thunk twice has no observable effect: both
evaluations are guaranteed to come up with the same result.  Thus, we
should be able to keep updates to thunk unsynchronized, at the cost of
occasional duplication of work when two threads race to evaluate the
same thunk.

There are two details we must attend to.  The first detail is the fact
that a race can still be harmful, if we need to update a thunk in
multiple steps and the intermediate states are not valid.  In the case
of a thunk update, we need to both update the header and write down a
pointer to the result; if we update the header first, then the
intermediate state is not well-formed (the result field is empty); on
the other hand, if we update the result field first, we might clobber
important information in the thunk.  Instead, we leave a blank word in
every thunk where the result can be written in non-destructively, after
which the header of the closure can be changed.\footnote{Appropriate
write barriers must be added to ensure the CPU does not reorder these
instructions.}

\begin{verbatim}
word    step 1   step 2   step 3
   0     THUNK    THUNK      IND \ valid closure
   1         -   result   result /
   2   payload  payload  payload <- payload is slop
\end{verbatim}

The second detail is that some thunks take a long time to evaluate: we'd
like to avoid duplicating them.  Instead, when we start evaluating them,
we'd like other threads to block on us for the result.  This is
precisely what a \emph{black hole} (Section~\ref{sec:blackhole}) is for!
A thread seeking to evaluate the thunk will instead enter the black
hole; in a concurrent setting, we can place the thread on the blocked
queue of the other thread.  It is somewhat difficult to put a blocked
queue on the thunk itself (due to the lack of synchronization); instead,
GHC uses a per-thread list of black hole blockers which is traversed
every time a thread finishes updating a thunk.

In our original discussion, black holes were \emph{eagerly} written to
thunks upon evaluation, with the optimization that this could be
\emph{lazily} deferred for later.  If a black hole is written eagerly,
it is on the fast path of thunk updates, and we cannot use
synchronization.  We call these \emph{eager black holes} (also known as
\emph{grey holes}), which do not guarantee exclusivity.  Lazy blackholing is done more infrequently (when a thread is preempted), and thus
we can afford to use a CAS to implement them.\footnote{While multiple
    threads may have eagerly blackholed a thunk, we guarantee only one
    thread has lazily blackholed it.  If a thunk \emph{must not} be
duplicated, it can achieve this by forcing all of its callers to perform
lazy blackholing
(\texttt{noDuplicate\#}).  \texttt{unsafePerformIO} uses precisely
this mechanism in order to avoid duplication of IO effects.}

The upshot is that GHC is able to implement lazy evaluation without any
synchronization for most thunk updates, applying synchronization
\emph{only} when it is specifically necessary. The cost of this scheme
is low: a single extra field in thunk and a (rare) duplication of work
upon a race.

\subsection{Summary}

Haskell \emph{threads} are light-weight threads of execution which
multiplex onto multiple CPU cores.  Each core has a \emph{Haskell
execution context} which contains a scheduler for running these threads;
in order to handle FFI calls execution contexts can migrate from core to
core as necessary.  Threads are load balanced across execution contexts
by having execution contexts with work push threads to contexts which
don't.  Sparks are a simple way of utilizing idle cores when there is no
other real work to do.

By in large, all inter-thread communication in Haskell is explicit, thus
making it much easier to compile Haskell in a thread-safe way.
\emph{MVars}, \emph{asynchronous exceptions} and \emph{STM} are
explicitly used by Haskell code and can be efficiently implemented by
taking advantage of our representation of Haskell threads.  The basic
techniques by which these are synchronized are \emph{messages} and
\emph{white holes} (spinlocks).  The only implicit inter-thread
communication that occurs is lazy evaluation.  However, due to purity,
we can largely avoid synchronizing thunk updates, utilizing \emph{eager
blackholing} to reduce duplicate work, and synchronized \emph{lazy
blackholing} when duplication is unacceptable.

\subsection{Further reading}

We have said little about how to use the concurrency mechanisms described here. \XXX

\section{Lazy evaluation}


\subsection{Tagless: Lazy evaluation} \label{sec:lazy}

Unusually, the info table also contains \emph{code} for the object: any
object can be ``entered'' (by jumping to the code in the infotable) in
order to evaluate it.  This is perhaps the most important design
decision in STG: it prescribes a uniform data representation for both
fully evaluated data values (constructors and functions) and unevaluated
thunks---referred to collectively as \emph{closures}.  In the original
design of the STG, all code wanting to access a field in a data
structure first entered the closure and upon return would receive the
values of the data structure (a \emph{vectored return}).  The lack of
any check whether or not a thunk was evaluated or not (an obvious
alternative implementation strategy) is behind the ``tagless`` in STG's
name.

\SM{We don't do vectored returns, since GHC 4.0. Also
  return-in-registers, which you might also find menioned in the
  original STG paper, was removed in favour of unboxed tuples.}

\SM{Tagless is a bit debatable these days.  We have tags in pointers
  (pointer tagging).  Also the code pointer for a function closure is
  the entry code for the function (not the ``eval'' code).}

The fact that objects are always entered is an extremely helpful layer
of indirection which is used for a variety of purposes by the runtime.
For example, when a thunk finishes evaluation, we need to write back the
true value to the old memory location so the computation is not
repeated.  If the new value is larger than the thunk, it must be placed
in new heap memory and the old thunk replaced with an \emph{indirection}
pointer pointing to the new value.  An indirection object, then, simply
jumps to the code of the indirectee!  This flexibility is useful for a
number of other exceptional conditions, especially in the case of
concurrency.

Unfortunately, code compiled this way performs a lot of indirect jumps,
and modern branch prediction units on processors typically perform very
poorly under such control flow.  This might suggest that this data
representation is quite expensive and of little interest to implementors
of non-lazy languages.  Fortunately, modern GHC implements a
\emph{dynamic pointer tagging} scheme~\XXX{} which, in many cases,
eliminates the need to perform an indirect jump.  This scheme works by
using the lower order bits (two bits in a 32-bit machine, and three bits
in a 64-bit machine) in order to encode whether or not pointer is
already evaluated, and if it is, what the tag of the constructor is.
When the tag bit is absent, user code will enter the closure, as before.

We think that this representation is well worth considering even for
non-lazy-by-default languages.  In the case of data types with multiple
constructors, pointer tagging enables us to support efficient case
analysis of user-defined types \emph{without a memory
dereference}---only the tag bit must be consulted.  And, to reiterate,
the added flexibility the indirect jumps will greatly aid us later in
other components of the runtime system.

\subsection{Synchronization}

In previous sections, we asserted that execution of Haskell threads
could be made parallel, assuming that the compiled Haskell code was
thread-safe.  As any developer who has needed to write thread-safe code
can attest, this is a tall order!

Fortunately, much of this work is already done.  We have already seen various mechanisms
for explicit interthread communication, which are all designed with
concurrent execution in mind:  synchronizing them is a relatively simple
task.  Furthermore, the vast majority of Haskell code is pure, and needs no
synchronization.

However, there is one major feature we have to worry about: lazy
evaluation.  Recall that after a thunk has been evaluated, its value on
the heap must be replaced with the fully evaluated value. Multiple
threads could evaluate a thunk in parallel, so na\"ively, these writes
must be synchronized.  This is extremely costly: Haskell programs do a
lot of thunk updates!

Once again, our saving grace is purity: as thunks represent pure
computation, evaluating a thunk twice has no observable effect: both
evaluations are guaranteed to come up with the same result.  Thus, we
should be able to keep updates to thunk unsynchronized, at the cost of
occasional duplication of work when two threads race to evaluate the
same thunk.

A race can still be harmful, however, if we need to update a thunk in
multiple steps and the intermediate states are not valid.  In the case
of a thunk update, we need to both update the header and write down a
pointer to the result; if we update the header first, then the
intermediate state is not well-formed (the result field is empty); on
the other hand, if we update the result field first, we might clobber
important information in the thunk.  Instead, we leave a blank word in
every thunk where the result can be written in non-destructively, after
which the header of the closure can be changed.\footnote{Appropriate
write barriers must be added to ensure the CPU does not reorder these
instructions.}

\begin{verbatim}
word    step 1   step 2   step 3
   0     THUNK    THUNK      IND \ valid closure
   1         -   result   result /
   2   payload  payload  payload <- payload is slop
\end{verbatim}

\subsection{Black holes} \label{sec:blackhole}

Some thunks take a long time to evaluate: we'd like to avoid duplicating
their work.  What we would like is for threads to notice when someone is
working on a thunk, and wait for the result to become available.

The mechanism by which this is implemented is a \emph{black hole}, which
represents a thunk that is currently being evaluated.  A thunk can be
\emph{claimed} by overwriting it with a black hole.  Black holes were
originally proposed as a solution for a space leak that occured in some
cases of tail calls~\cite{Jones2008}, but they have found their utility
in a multithreaded setting.  Recall that a thread wishing to evaluate
a thunk jumps to the entry code; the entry code of a black hole
places a thread on the blocked queue of the owner the black hole, to
be woken up when the thunk has been evaluated.\footnote{It is somewhat difficult
to put a blocked queue on the thunk itself (due to the lack of
synchronization); instead, GHC uses a per-thread list of black hole
blockers which is traversed every time a thread finishes updating a
thunk.}

There are two times when a black hole can be written: it can be \emph{eagerly}
written as soon as a thunk is evaluated, or it can be \emph{lazily} deferred
for when a thread has gotten descheduled (and thus the thunk was taking
a long time to evaluate.)  If a black hole is written eagerly,
it is on the fast path of thunk updates, and we cannot use
synchronization.  We call these \emph{eager black holes} (also known as
\emph{grey holes}), which do not guarantee exclusivity.  Lazy blackholing is done more infrequently, and thus
we can afford to use a CAS to implement them.\footnote{While multiple
    threads may have eagerly blackholed a thunk, we guarantee only one
    thread has lazily blackholed it.  If a thunk \emph{must not} be
duplicated, it can achieve this by forcing all of its callers to perform
lazy blackholing
(\texttt{noDuplicate\#}).  \texttt{unsafePerformIO} uses precisely
this mechanism in order to avoid duplication of IO effects.}

The upshot is that GHC is able to implement lazy evaluation without any
synchronization for most thunk updates, applying synchronization
\emph{only} when it is specifically necessary. The cost of this scheme
is low: a single extra field in thunk and a (rare) duplication of work
upon a race.


\section{Acknowledgements}

Alexander Chernyakhovsky for helpful discussions.

\bibliographystyle{jfp}
\bibliography{paper}

\end{document}
