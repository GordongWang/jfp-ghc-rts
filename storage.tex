\section{Storage}

An essential component of a runtime system for any high-level
programming language is the garbage collector, which is responsible
identifying and reclaiming memory from objects which are no longer in
use by the program.

The presence of a garbage collector imposes important restrictions on
the design of the runtime system.  Additionally, the implementation of a
garbage collector is greatly complicated by its performance
considerations: the garbage collector affects the performance of all
programs.  Fortunately, the design of the garbage collector itself has
very few dependencies on the rest of the runtime system (the most
interesting thing about our setting is the unusual data life cycle that
accompanies functional programs, which we will discuss in Section~\XXX).

Let us assume, then, that we are building a runtime for a programming
language with managed memory.  What do we need?

\subsection{Blocks}

The very first consideration is this: ``Where is the memory coming
from?''  The conventional design is to request a large, contiguous block
of memory from the operating system and use it as the heap.  However,
this scheme is fairly inflexible, especially when multiple heaps are
necessary (as might be the case in a generational garbage collector.)

Instead, GHC utilizes a \emph{block-structured heap}~\cite{Marlow:2008:PGG:1375634.1375637}.
The basic idea is to divide the heap into fixed-size $B$-byte blocks,
where $B$ is a power of two: blocks are then linked together in order to
provide memory for the heap.  GHC uses 4kb blocks, but this is an easily
adjustable constant.  With blocks, we can allocate and dynamically
resize regions of memory at will.  There are some other benefits as well:

\begin{enumerate}
    \item Large objects (e.g. close to block size) do not have to be copied from one region to
        another; instead, the block they reside in can be relinked from
        one region to another.\footnote{Of course, this requires \emph{only}
        one object to live in a block, which can result in fragmentation.
        However, empirically this does not seem to have caused much of a problem.}
    \item Blocks make it easy to provide heap memory in contexts where it is
        not possible to perform garbage collection.  The primary example of this
        in GHC is the GMP arbitrary-precision arithmetic library, which requires
        the ability to allocate memory for its internal computation.  If the heap
        was contiguous, we would be left with no recourse if the heap ran out
        of memory while GMP code was executing, since there is no way to resize
        the heap without simultaneously executing a garbage collection, and
        precise garbage collection cannot be performed safely while running C code.
    \item Free memory can be recycled quickly, since a free block can be quickly
        used somewhere else.
\end{enumerate}

One reason why this scheme works so well is that most objects on the
heap are much smaller than the block size; handling these cases is very
simple.  When an object is larger than a block size, it needs to be
placed into a \emph{block group} of contiguous blocks---which in turn
need to be handled with some care to avoid fragmentation.  The blocks
themselves are provided by the operating system in units called
\emph{megablocks}, which in the current implementation are 1Mb in size.

Finally, each block has associated with a \emph{block descriptor}, which
contains information about the block such as what generation it belongs to, how full it is, what block
group it is part of, etc.  An obvious place to put the block descriptor
is at the beginning of a block, but this runs into problems when the block
is the member of a block group (the memory must be contiguous!)
Thus, the descriptors of blocks of a megablock are instead organized together
in the first block of a megablock; some care is taken to ensure that the
runtime can efficiently compute the block descriptor of any given block---an
interested reader can find the details at \verb|http://hackage.haskell.org/trac/ghc/wiki/Commentary/Rts/Storage/BlockAlloc|
\XXX


The next classic problem a garbage collector encounters is the question:
``What is a pointer?''  The garbage collector must be able to answer
this question in order to follow the pointers and get an accurate
picture of what memory is in use.  The classic solution to this problem
is to require data on the heap to be \emph{boxed}: every object has
a header which says what type of object it is and what the pointer fields
in the object are.

\subsection{Generational garbage collection}

\subsection{Parallel garbage collection}

\subsection{Summary}
