\section{Concurrency and parallelism}

We now turn our attention to the implementation of
concurrency~\cite{PeytonJones:1996:CH:237721.237794} and
parallelism~\cite{Harris:2005:HSM:1088348.1088354} in the GHC runtime.
\Red{Something here; maybe the old saw about the difference between
concurrency and parallelism, or maybe something about why Haskell's
concurrency/parallelism support is so cool.}

\Red{Some motivation here}

%   The implementation of many of the features in this section can be
%   divided into two parts: an implementation in the uniprocessor case
%   (concurrency), and an implementation in the multiprocessor case
%   (parallelism).

\subsection{Threads}

Concurrent Haskell~\cite{PeytonJones:1996:CH:237721.237794} exposes the
abstraction of a \emph{Haskell thread}, which is a concurrent thread of
execution.  Given that operating systems also native threads, the
essential question for an implementation of Haskell threads is this:
what is the correspondence between Haskell threads and OS threads?

A simple scheme is to require a \emph{one-to-one} mapping, an approach
taken by many other languages with multithreading support.
Unfortunately, this approach is expensive: the GHC runtime would like to
support thousands of threads, which is supported poorly by most
underlying operating systems.  Furthermore, in the case that a program
is run with only one operating system thread (e.g. the code in question
is not thread safe), only one thread of execution is supported: all
concurrency is lost, and must be implemented in userspace, as is the
case with many asynchronous IO libraries.

Instead, we must \emph{multiplex} multiple Haskell threads onto a single
OS thread.  This requires two adjustments: first, we must be able to
induce compiled Haskell code to yield to a scheduler, so that another
Haskell thread can be run.  It is very easy for a thread to
cooperatively suspend itself: all it needs to do is ask for a garbage
collection; thus, we can preempt a thread by setting its heap limit to
zero, triggering a faux ``heap overflow'' that transfers control to the
scheduler.\footnote{This can have problems when a thread is in a tight,
    non-allocating loop; however, better concurrency in such cases can
be achieved by forcing the compiler to emit heap checks at all function
prologues, even when no allocation is necessary.}
Second, we must save the thread's context, e.g. the stack pointer and
the errno flag, so that it can be restored when the thread is
rescheduled.  Because the compiled code is responsible for yielding to
the scheduler, it can be very efficient about how much other processor
state must be saved; while general purpose C-based coroutine libraries
must save all registers,~\XXX{} a Haskell thread will usually only need
one or two registers to be saved.\footnote{This is what the fast path \texttt{stg\_gc\_*} functions are for.}

Once we have a way of suspending and resuming threads, the scheduler
loop is quite simple: maintain a run queue of threads, and repeatedly:

\begin{enumerate}
    \item Pop the next thread off of the run queue,
    \item Run the thread until it yields or is preempted,
    \item Check why the thread exited:
        \begin{itemize}
            \item If it ran out of heap, call the GC and then run the thread again;
            \item If it ran out of stack, enlarge the stack and then run the thread again;
            \item If it was preempted, the thread is pushed to the end of the queue;
            \item If the thread exited, continue.
        \end{itemize}
\end{enumerate}

Once we have a scheduler loop, we have a simple way of running multiple
Haskell threads on a \emph{single} OS thread, even when the Haskell
threads are not thread-safe.

\subsection{Foreign function interface}

While multiplexing is a very attractive solution for supporting
lightweight concurrency, there are some places when the illusion of ``a
Haskell thread is simply an operating system thread, but cheaper''
breaks down.  These cases are especially prevalent when considered with
the \emph{foreign function interface} (FFI)~\cite{Marlow04extendingthe},
which permits Haskell code to call out to foreign code not compiled by
GHC, and vice versa.  One particular problem is as follows: what if an
FFI call blocks?  As our concurrency is purely cooperative, if the FFI
call refuses to return to the scheduler, the execution of all other
threads will grind to a halt.

The upshot is that we need to decouple OS threads into two parts: the OS
thread itself (called a \emph{Task} in GHC terminology), and the Haskell
execution context (called a \emph{Capability}).  The Haskell execution
context can be thought of as the scheduler loop and is responsible for
the contents of the run queue; when executing it is owned by the
particular OS thread which is running it.  A single-threaded Haskell
program, there would only be one capability---in this case, the
capability can be thought as the lock on the Haskell runtime.  This
decoupling allows a capability to be run on different operating system
threads as necessary: now a blocking FFI call can be managed by
releasing the capability before making the FFI call: if there is another
idle \emph{worker thread}, it can pick up the now free capability and
continue running Haskell code.  It is preferable to move the Haskell
execution context, rather than the FFI call, to another thread, as it
avoids an OS level thread context switch.\footnote{To see this, consider
the execution from the perspective of the OS thread, which executes
an FFI call by releasing a lock on an object (the capability),
running some C code, and then re-acquiring a lock on the
capability.}

Thus, we have to modify the scheduler loop as follows:

\begin{enumerate}
    \item Check if we need to yield the capability to some other OS thread, e.g. if an FFI call has just finished,
    \item \emph{Run as before.}
\end{enumerate}

Another problem that occurs is when the FFI code maintains some thread
local state.  When we pass capabilities around OS threads, we make no
guarantee that any given FFI call will be performed on the same OS
thread: this can cause problems for FFI code of this type.  To
accomodate this, Haskell introduces the concept of a \emph{bound
thread}, which is a thread which is guaranteed to run on the same
operating system thread whenever it is scheduled.  GHC refers to these
as \emph{in-calls}, because any external code which calls into GHC will
always create a bound thread, as its Haskell code may make an FFI call
which must be scheduled on the same original OS thread.  We must modify
the scheduler loop some more:

\begin{enumerate}
    \item \emph{Check if we need to yield the capability to some other OS thread, e.g. if an FFI call has just finished,}
    \item \emph{Pop the next thread off of the run queue,}
    \item Check if the thread is bound:
        \begin{itemize}
            \item If the thread is bound but is already scheduled on the OS thread, proceed.
            \item If the thread is bound but on the wrong OS thread, give the capability to the correct task.
            \item If the thread is not bound but this OS thread is bound, give up the capability, so that any capability that truly needs this OS thread will be able to get it.
        \end{itemize}
    \item \emph{Run as before.}
\end{enumerate}

The movement of capabilities from task to task is somewhat intricate,
but it is necessary to properly support FFI calls, and imposes very
little overhead when no FFI calls are being made.

\subsection{Load balancing}

Assuming that the compiled Haskell code is thread safe~\XXX, it is now
very simple to have the scheduler loop run on multiple OS threads:
allocate multiple capabilities!  Each OS thread in possession of a capability
runs the scheduler loop, and everything works the way you'd expect.

There is one primary design choice: should each capability have its own
run queue, or should there be a single global run queue?  A global run
queue avoids the need for any sort of load balancing, but requires
synchronization and makes it difficult to keep a Haskell thread on one
core, destroying data locality.  With separate run queues, however,
threads must be load balanced, lest one capability accumulate too many
threads while the other capabilities idle.

The very simple load balancing scheme GHC uses is as follows: when a capability
runs out of threads to run, it suspends itself (releasing its lock) and
waits on a condition variable.  When a capability has too many threads
to run (it checks each iteration of its schedule loop), it takes out
locks on as many idle capabilities as it can, and then pushes its excess
threads onto their run queues.  Finally, it releases the locks and
signals on each idle capability's condition variable that they can
continue running.  The benefit of this scheme is that the run queues are
kept completely unsynchronized. \Red{Why isn't a work stealing queue used here?}

\subsection{Messages and whiteholes}

Up until now, we focused purely on how to ensure multiple threads could
execute concurrently, without considering the other essential aspect of
concurrency: synchronized communication.  Haskell offers a variety of
ways for Haskell threads to interact with each other (MVars,
asynchronous exceptions, STM, even lazy evaluation).  When these features
are implemented on a uniprocessor, they can be implemented directly;
in a multithreaded setting, some synchronization is necessary.

Under the hood, the GHC runtime has two primary ways of synchronizing:
\emph{messages} and \emph{whiteholes} (effectively spinlocks).  The
runtime makes very sparing use of OS level condition variables and
mutexes, since these operations tend to be expensive.

GHC uses a very simple message passing architecture to pass messages
between capabilities.  A capability sends a message by performing the following steps:

\begin{enumerate}
    \item Allocate a message object on the heap with the appropriate data;
    \item Take out a lock on the message inbox of the destination capability;
    \item Append the message onto the inbox;
    \item Interrupt the capability, using the same mechanism as the context switch timer (setting the heap limit to zero);
    \item Release the lock.
\end{enumerate}

This allows the message to be handled by the destination capability at
its convenience, i.e. after the running Haskell code yields and we
return to the scheduling loop.  The primary benefit of message passing
is it removes the need to synchronize any of the non-local state that a
capability \emph{might} want to modify. \Red{Why don't the message
inboxes use spinlocks?}

When a closure on the heap needs to be locked, e.g. because it is a
mutable cell, GHC uses a very simple spinlock, attempting to replace the
header of a closure with a \emph{whitehole} header.  If another thread
enters the closure, they will spinlock until the original header is
replaced. \Red{An example of the entry code indirection helping out!}
Spinlocks are great for locking closures, since the critical regions
they protect are very short, and it would be expensive to allocate a
mutex for every closure that needed one.

\subsection{MVar}

We now describe how to implement MVars, the simplest form of
synchronization available to Haskell threads.  An MVar is a mutable
location that may be empty.  There are two operations which operate on
an MVar: \verb|takeMVar|, which blocks until the location is non-empty,
then reads and returns the value, leaving the location empty, and
\verb|putMVar|, which dually blocks until the location is empty, then
writes its value into location.

The blocking behavior is the most interesting aspect of MVars:
ordinarily, one would have to implement this using a condition variable.
However, because our Haskell threads are not operating system threads,
we can do something much more lightweight: when a thread realizes it
needs to block, it adds itself to a \emph{blocked threads queue}
corresponding to the MVar, and then removes itself from the run queue.
When another thread fills in the MVar, it can check if there is a thread
on the blocked list and wake it up immediately.  In a multithreaded
setting, the only necessary synchronization is a spinlock for the queue
manipulation and a wakeup message to the capability which owns the
unblocked thread.

This scheme has a number of good properties.  First, it allows us
to implement efficient \emph{single wake-up} on MVars, where only one of
the blocking threads is permitted to proceed. Second, using a FIFO
queue, we can offer a fairness guarantee, which is that no thread
remains blocked on an MVar indefinitely unless another thread holds the
MVar indefinitely.  Finally, because threads are garbage collected
objects, if the MVar a thread is blocking on becomes unreachable,
\emph{so does the thread.}  Thus, in some cases, we can tell when
a blocked thread will never make any further progress, and terminate it.

\subsection{Asynchronous exceptions}

MVars are a cooperative form of communication, where a thread must
explicitly opt-in to receive messages.  Asynchronous
exceptions~\cite{Marlow:2001:AEH:378795.378858}, on the other hand
permit threads to asynchronously raise an exception in another thread
without its cooperation.  Asynchronous exceptions are much more
difficult to program than their synchronous brethren: as the signal can
occur at any point in a program's execution, the program must be careful
to register handlers which ensure that any locks are released and
invariants are preserved.  In \emph{pure} functional programs, this requirement
is easier to fulfill, as pure code can always be safely aborted.  When
asynchronous exceptions are available, however, they are useful
in a variety of cases, including timing out long running computation,
aborting speculative computation and handling user interrupts.

With messages, it is very simple to trigger an asynchronous exception:
simply send a throw-to message to the capability which owns the thread
to be interrupted.  Once the thread is preempted, the scheduler induces
an exception and walks up the stack, much in the same way as a normal
exception would be handled.

However, there is a big difference between how \Red{Continue me!}

\subsection{STM}

\subsection{Sparks}

Up until now, the threads we have discussed were explicitly asked for
by the user.  In some cases, the user will only have one thread (because
their application does not need to be concurrent), in which case extra
cores will be unused.  Is there any way we can take advantage of extra
capacity?

Haskell takes advantage of pure lazy evaluation to offer \emph{sparks},
which are extremely lightweight threads with the express purpose of
evaluating a thunk to head normal form.  Because thunks do not have
side effects, it is safe to evaluate them speculatively, or to not
evaluate them at all; if they are truly needed they will get forced
appropriately.

Since sparks are even lighter-weight than threads, they are represented
separately and stored in \emph{spark pools}.  When a capability decides
it has no work to do, it creates a \emph{spark thread}, which repeatedly
attempts to retrieve a spark from the capability's spark pool and
evaluate it.  If the capability receives any real work (e.g. a thread
unblocks), then it responds to the spark threads request that there are no
more sparks to evaluate, so that it exits.

It is relatively important to ensure sparks can be load balanced across
capabilities, as the capability that is most likely to be generating
sparks is one that also is doing real work.  Sparks are balanced using
bounded work-stealing queues~\XXX; empirically, it seems that using a FIFO
ordering (so the oldest sparks are stolen first) is best, although this
requires more synchronization on our implementation of work-stealing
queues.

\Red{Comment about interaction with GC}

\subsection{Synchronization on the heap}

In previous sections, we asserted that execution of Haskell threads
could be made parallel, assuming that the compiled Haskell code was
thread-safe.  As any developer who has needed to write thread-safe code
can attest, this is a tall order!  However, in Haskell we have a leg up:
the vast majority of code written is pure and does not perform explicit
mutation, so we do not mind paying the costs of explicitly synchronizing
other mutable state (as is the case for MVars). However, there is one
major performance bottleneck that we have to worry about: lazy
evaluation!  Recall that after a thunk has been evaluated, its value on
the heap must be replaced with the fully evaluated value.  As multiple
threads could evaluate a thunk in parallel, these writes must be
synchronized. A poorly implemented synchronization scheme could be
extremely costly.

\Red{describe the scheme in \cite{Harris:2005:HSM:1088348.1088354}}

\subsection{Summary}

\subsection{Further reading}
